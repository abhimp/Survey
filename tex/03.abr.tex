\section{Adaptive Bitrate (ABR) Algorithm}
Efficient ABR algorithm is required to get high QoE. However, it is extremely challenging task as the video can be played in any devices including smartphone, low end screen to very high end gaming console, TV. The devices can use a different variety of links to connect to the Internet including cellular, WiFi, broadband. User can also change there connection in while playing the video. From the inception of the DASH, researcher have user different method to design ABR algorithm. This methods can be {\tt buffer based}, {\tt throughput based}, {\tt hybrid} or {\tt other}. We discuss these algorithm in detail.

\subsection{Throughput based ABR algorithm}
Any ABR's primary job is to play video without rebuffering. To acheive this simple task, initially ABR algorithm used to start with lowest bitrate to quickly start the video and improve quality in subsequent segment until it detect some congestion or loss \cite{5677508,10.1145/1943552.1943575}. While this technique avoid the rebuffering, it change video quality for almost for all the segment. Microsoft start using little more conservative solution in \cite{10.1145/1943552.1943574} while adapting bitrate based on the network quality. These algorithms have mitigated the rebuffering by matching the video bitrate with the observer network bitrate. However they have failed to considered the fact that the frequent change in quality can be a issue. These algorithms has goal to improve the quality and reduce the rebuffering. Later on, ABR algorithm are design to maintain the QoE. Few such algorithms are:

\subsubsection{QDASH\cite{10.1145/2155555.2155558}}
Mok \etal\ design the QDASH\cite{10.1145/2155555.2155558}, one of the early ABR algorithms with the awarness of QoE, consider the abrupt quality change is not acceptable. The algorithm have two parts, a) ABR algorithm and b) Bandwidth measurement tool. They proposed a special module to run at the streaming server which measure the client bandwidth. The ABR algorithm explicitly connect the measurement module and get in the bandwidth information. With the bandwidth information, QDASH find the most suitable bitrate. However, QDASH does not switch to most suitable bitrate immediately, rather it switch to intermediate bit rate to increase the QoE.

\subsubsection{FESTIVE}
FESTIVE\cite{10.1145/2413176.2413189} is a DASH based video streaming system that designed to provide fairness between players that share bottleneck in a stable and efficient way. Jiang \etal\ show the measured throughput may not be correct, and the available bottleneck bandwidth might be underutilized due to the scheduling scheme. So, they suggested a method including three steps. These steps are i) estimate the available bandwidth as the harmonic mean of last 20 throughput measurement, ii) never to jump bitrate, increment or decrement should not go beyond 1 level, and iii) randomized the segment scheduling so that all the player get a fair share of the bottle link.

\subsubsection{ELASTIC\cite{6691442}}
All the client side adaptive algorithm generate on-off traffic which lead to unfairness among player. Cicco \etal\ designed ELASTIC\cite{6691442}, a client-side controller that does not generate on-off traffic pattern. ELASTIC select video bitrate for a segment in such a way so that it finishes downloading the segment at the time when next segment needs to be downloaded. So, there is not requirement of the on-off traffic generation, and traffic can be fair as underlying TCP is itself fair for long running system.

\subsubsection{Presto\cite{7249417}}
Zhang \etal\ design Presto\cite{7249417} to provide fairness among player when players are playing from multiple servers. THey argue that a player might consuming more bandwidth of the bottleneck by running more number of parallel flows. So, they provide a mechanism to restrict bitrate of a player with more flows and improve quality. Presto exploit it the fact that a user get annoyed with drastic bit drop but not with drastic bitrate increase\cite{10.1145/2018602.2018611}. It also suspend some flows for sometime so bandwidth utilization stabilize for the time and resume later.

\subsubsection{SQUAD\cite{10.1145/2910017.2910593}}
SQUAD\cite{10.1145/2910017.2910593} is a spectrum\cite{1386243} based quality adaptation technique for DASH based video streaming system. Wang \etal\ assumes spectrum as the variation in the bitrate around some $N$ number of segments. The proposed system start with providing a new method measure the throughput. Normally throughput are measured when segments have been downloaded and estimations are made based on the measured throughput. Authors have pointed out that this technique may not be correct as the size of the segments are different and thus it might take different time to download segments even when the bandwidth is same. So, SQUAD suggested to do running average of throughput over the course of the segment downloading. Then it proposed to set the bitrate of few initial segment to the lowest bitrate available and after initial set of segment, set bitrate in such a way so that the bitrate variation in a window i.e. SPECTRUM is lowest.

\subsubsection{CS2P\cite{10.1145/2934872.2934898}}



\subsection{Buffer Based}
Buffer based bitrate adaptation technique is alternative technique of througput based where the player does not need to estimate the throughput to adapt the playback quality. As througput estimation is a complex and most of the time inaccurate due to the nature of network dynamics. Also, the middle catching devices including proxy make it more difficult to estimate real througput. Buffer based ABR algorithms do not suffers from similar problem. In the past decade, researcher have designed several buffer based ABR algorithms. We are going discuss few of those algorithms.

\subsubsection{Buffer based rate adaption\cite{10.1145/2619239.2626296,10.1145/2398776.2398800,10.1145/2491172.2491179}}
Huang \etal(\cite{10.1145/2619239.2626296,10.1145/2398776.2398800,10.1145/2491172.2491179}) describe a bitrate adaptation mechanism solely based on the playback buffer level. The author suggested that the playback buffer is a function of bitrate and network throughput, thus player buffer level can be a indication whether to change bitrate or not. Authors use a rate to buffer map to calculate the bitrate for a give buffer. They use a algorithm to remove any fine boundary between to bitrate to avoid frequent bitrate flactuation.

\subsubsection{Threshold driven buffer based adaptation\cite{6229732}}
Miller \etal\ designed a buffer based algorithm in \cite{6229732} as an remedy for the measurement problem of througput driven algorithms. They uses 3 buffer levels $B_{min}$, $B_{low}$ and $B_{max}$ ($0 < B_{min} < B_{low} < B_{max}$) as threshold to decide the bitrate switch. The algorithm aims to keep the buffer level to close to mean of $B_{low}$ and $B_{max}$. It decrease the bitrate if buffer level goes below $B_{low}$, drops to lowest if it goes below $B_{min}$ and increase if it goes beyond $B_{max}$.

\subsubsection{SABRE\cite{10.1145/2483977.2484004}}
SABRE is designed to mitigate the buffer bloat effect of TCP, which might cause queuing delay upto few seconds in case of HTTP streaming. Mansy \etal(\cite{10.1145/2483977.2484004}) suggested to use HTTP pipeline where multiple HTTP-request can be pushed together. At the client side, make {\tt recv} call in non-aggressive way to reduce the queuing delay. Due to it's paced {\tt recv} call, it is impossible measure throughput correctly. So, the change the pacing of {\tt recv} call based on the playback buffer. The player, change the bitrate as per the observed download rate.

\subsubsection{SARA\cite{7247436}}
SARA\cite{7247436} is segment aware rate adaption technique developed by Juluri \etal. It is a mixed adaptive system where the playback buffer is used to decide how the bitrate should change and the throughput is used to decide apropriate bitrate. It have four bitrate adaptation scheme based on four buffer thresholds. These are a) slowstart: when buffer ($B_{curr}$) very low ($B_{curr}<I$), lowest bitrate is selected, b) additive increase: when buffer is moderated ($B_{\alpha} \le B_{curr} > B_{\beta}$), it care fully increase the bitrate and c) aggressive bitrate switching: if the buffer is high enough ($B_{\beta} \le B_{curr} \ge B_{max}$, bitrate can jump without any effect and d) delayed download: if the buffer is saturated ($B_{max} < B_{curr}$), download have to wait for buffer goes down.

\subsubsection{Buffer Occupancy based Lyapunov Algorithm (BOLA)\cite{7524428}}
BOLA\cite{7524428} treat the the ABR streaming as optimization problem where average bitrate need to be maximize and rebuffering needs to be minimize. Spiteri \etal\ solves the problem using Lyapunov optimization and provides theoritical guaranti on the QoE. BOLA does not require to compute the througput, rather it solely based on playout buffer.

\subsubsection{ABMA+\cite{10.1145/2910017.2910596}}
Beben \etal\ proposes ABMA+\cite{10.1145/2910017.2910596} which predict the video freezing probablity for each representation and representation that statisfy the predefined threshold. It continously monitor the segment download time and using time characteristics and the precomputed buffer-map it decides best bitrate for smooth playback.

\subsubsection{mDASH\cite{7393865}}
mDASH is Markov-decision based bitrate adaptation algorithm proposed by Zhou \etal\ in \cite{7393865}. Author uses a heterogeous and time varying Markov model to estimate the future throughput. \red{{\bf Difficult to understand, push it to other section.}}

\subsection{Hybrid ABR algorithms}
There are algorithm which consider both buffer and throughput to adapt the bitrate. These algorithm can be further categorised in control system based and without control system based. We will discuss few of them here.

\subsubsection{Smooth rate adaptation\cite{10.1145/2413176.2413190,6694183}}
The authors of \cite{10.1145/2413176.2413190} and \cite{6694183} proposes a method involving a control loop and measured throughput and playback buffer. The control loops start with playback buffer and a reference buffer and difference between the these two is used to predict the throughput. Then the predicted throughput is used to determine the video bitrate. The video bitrate and the real throughput lead to playback buffer occupancy which is further used in the control loop.

\subsubsection{Panda\cite{140405}}
Li \etal\ found that most of the widely deployed DASH like streaming suffers from bitrate fluctuation when multiple player shares a bottleneck. They also found that the fluctuation depends on various parameter including the number of player, players start time, background traffic. While digging more they that all the technique assume that the measure throughput is fair and use that directly. However, it is not true and due to this fundamental mistake player lost the capability of over-coming buffer underrun. Li \etal\ suggested to use probe the network to find the available bandwidth. PANDA uses TCP like AIMD scheme for rate-adaptation. However, it only use the probe per chunk instead of per RTT.

\subsubsection{Model predictive control algorithm\cite{10.1145/2785956.2787486,10.1145/2670518.2673877}}
Model predictive control (MPC) algorithm is proposed by Yin \etal\ in their paper \cite{10.1145/2785956.2787486,10.1145/2670518.2673877} as superios ABR algorithm than the existing algorithms by optimally combine the throughput and buffer occupancy information. Author formulate the QoE optimization problem as a stochastic optimal control problem and tries to solve it using MPC. They formulate the bitrate selection as a function of three components a) predicted future bitrate, b) buffer occupancy and c) available bitrates. The basic MPC algorithm is made up with three steps: \\
a) Predict: Although the future throughput prediction in difficult most of the time inaccurate, they assumes that there is a way to do it with acceptable errors. In throughput prediction step they can use any third party algorithm to predict the throughput.\\
b) Optimize: In this phase, the MPC search for the bitrate for next $N$ segments with predicted throughput and calculated buffer occupancy and choose the best one. The problem can be solved using {\tt CPLEX} solver.\\
c) Apply: Once bitrate is selected, it should start downloading the segment.\\
The optimization step is computationally heavy and it have to perform before downloading each segment. So, authors have proposed a alternative to do it fast by using lightweight combinatorials.

\subsection{(Machine) Learning based ABR algorithms}
Till now we discuss various algorithm that uses parameter like instantenious buffer occupancy and throughput to decide the bitrate of the bitrate. However, researcher have also developed learning based algorithm to solve the problem. In this section we discuss some those algorithms.

\subsubsection{Multi-agent Q-Learning}
Petrangeli \etal\ proposes Multi-agent Q-Learning\cite{6838245}, a reinforcement-learning based ABR algorithm. The algorithm does not change the client side architecture, however, add a co-ordinating proxy server in between the server and client. The proxy server collect and aggreate the reward from all the players. It then compute the global global signal from the reward and broadcast to the players. The global signal is use to compute the Homo Egualis \cite{10.5555/1402298.1402344} like reward which is used to provide fairness among the players. With this Homo Egualis reward players are able compute the local reward which in turn leads to selection of video quality with the help of throughput, segment lengths and quality levels.

\subsubsection{RL based Online Learning Adaptation}
Chiariotti \etal\ formulate the Adaptation problem as Markov Decision Process (MDP)\cite{P-1066} optimization problem\cite{10.1145/2910017.2910603}. Authors have find out the future state of the process solely depends on the present state. Here state of the system defined as quality level, available throughput, size of the segment and the playback buffer. The state transition happened based on the action taken in current state based on the action, which is essentially the bitrate, and reward calculated based on the previous state, action taken in previous state and current state. Then they use online reinforcement learning to solved the problem to make the best possible decision.

\subsubsection{D-DASH \cite{8048013}}
D-DASH\cite{8048013} is an deep-neural network based ABR solution. Like \cite{10.1145/2910017.2910603}, Gadaleta \etal\ formulate the problem as MDP optimization. However, author suggested Q-Learning based deep-neural network solution instead of RL-based solution as it compact and fast.

\subsubsection{Pensieve}
Mao \etal\ designed Pensieve\cite{10.1145/3098822.3098843} to solve the bitrate selection problem using Recurrent Neural Network. The Pensieve treat multiple parameter including buffer occupancy, download history, playback time as the current state and bitrate selection is action to move to next state. Pensieve use the QoE as the reward for state transition. The aim here is to increase the reward i.e. QoE by taking accurate action. Pensieve selection action based on the policy, which a probability distribution of state and action pair. As there are intractably many such pair exists, pensieve uses neural network to represent the policy with manageable parameters. It uses actor-critic network to train the policy with policy gradient method\cite{sutton1999policy}. It also uses the A3C\cite{10.5555/3045390.3045594} algorithm, a actor-critic method involvin two networks, to train its model.\\
Pensieve proposes a simulation based method to train the network fast and then the train model to used in real playback system. To make the training process even faster, they proposes asynchronous parallel training involving primary and secondary networks.

\subsubsection{Oboe}
Oboe\cite{10.1145/3230543.3230558} is another learning based ABR technology devised by Akhtar \etal. While Oboe does not directly provide any algorithm to select bitrate for each segment, it tune the confguration of other ABR algorithms including MPC\cite{}, BOLA\cite{} and Penseive\cite{} based on the network state. Oboe precompute the apropriate configuration set for any algorithm based on the different network state using reinfocement learning and apply the learned model online to tune the configuration to provide best possible outcome by a ABR algorithm.

\subsection{Mixed}
\red{{\bf This is not a section or subsection. I am going to accumulate the different recent ABR algorithms here so that I can categorized them later.}}



\subsubsection{Favor}
Fine-grained Video Rate Adaptation or Favor\cite{10.1145/3204949.3204957} proposed by the He \etal\ to optimize the non-conventional parameters like frame-dropping rate, playback speed along with convention parameter bitrate to optimize DASH based player beyond the optimization horizon. The author suggested that viewer cannot notice upto 35\% unneccessary frames can be dropped as well as upto 25\% reduction in playback speed. Although these changes non-conventional, it can allow player to cope with the throughput reduction as frame reduction cause the segment size reduction and playback rate reduction gives more time to download segment. Favor also suggest a framework for 360$^{\circ}$ video by tiling and optimizing individual tiles.

\subsubsection{LiveNAS}
