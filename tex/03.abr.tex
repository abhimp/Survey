\section{Adaptive Bitrate (ABR) Algorithm}
Efficient ABR algorithm is required to get high QoE. However, it is extremely challenging task as the video can be played in any devices including smartphone, low end screen to very high end gaming console, TV. The devices can use a different variety of links to connect to the Internet including cellular, WiFi, broadband. User can also change there connection in while playing the video. From the inception of the DASH, researcher have user different method to design ABR algorithm. This methods can be {\tt buffer based}, {\tt throughput based}, {\tt hybrid} or {\tt other}. We discuss these algorithm in detail.

\subsection{Throughput based ABR algorithm}
Any ABR's primary job is to play video without rebuffering. To acheive this simple task, initially ABR algorithm used to start with lowest bitrate to quickly start the video and improve quality in subsequent segment until it detect some congestion or loss \cite{5677508,10.1145/1943552.1943575}. While this technique avoid the rebuffering, it change video quality for almost for all the segment. Microsoft start using little more conservative solution in \cite{10.1145/1943552.1943574} while adapting bitrate based on the network quality. These algorithms have mitigated the rebuffering by matching the video bitrate with the observer network bitrate. However they have failed to considered the fact that the frequent change in quality can be a issue. These algorithms has goal to improve the quality and reduce the rebuffering. Later on, ABR algorithm are design to maintain the QoE. Few such algorithms are:

\subsubsection{QDASH\cite{10.1145/2155555.2155558}}
Mok \etal\ design the QDASH\cite{10.1145/2155555.2155558}, one of the early ABR algorithms with the awarness of QoE, consider the abrupt quality change is not acceptable. The algorithm have two parts, a) ABR algorithm and b) Bandwidth measurement tool. They proposed a special module to run at the streaming server which measure the client bandwidth. The ABR algorithm explicitly connect the measurement module and get in the bandwidth information. With the bandwidth information, QDASH find the most suitable bitrate. However, QDASH does not switch to most suitable bitrate immediately, rather it switch to intermediate bit rate to increase the QoE.

\subsubsection{FESTIVE\cite{10.1145/2413176.2413189}}

\subsubsection{}

\subsection{Buffer Based}

\subsubsection{\cite{10.1145/2619239.2626296,10.1145/2398776.2398800,10.1145/2491172.2491179}}

\subsubsection{\cite{6229732}}

\subsubsection{\cite{7524428}}

\subsubsection{\cite{10.1145/2910017.2910596}}

\subsection{Hybrid/control theory}

\subsubsection{\cite{10.1145/2413176.2413190,6694183}}

\subsubsection{\cite{140405}}

\subsubsection{\cite{10.1145/2910017.2910593}}

\subsubsection{\cite{10.1145/2483977.2484004}}

\subsubsection{\cite{10.1145/2785956.2787486,10.1145/2670518.2673877}}

\subsubsection{\cite{6691442}}