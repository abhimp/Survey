
\section{Live}
DASH like video streaming system is widely used as it superior ot other system and live video streaming in not an exception. Although interactive video session are difficult to support via DASH due to the rigid latency bound, non-interactive video streaming are supported. Static (or VoD) and live video streaming are almost same expect one part that video segment are not available in the system. All the segment need to capture, processed and distribute to nearest server before it can be served to the client. This small difference invite lot of innovation towards the DASH based live streaming. We will discuss few of those innovation in these section.

\subsection{Crowdsourced Live Streaming}
Live video streaming require live transcoding into multiple different quality version as well to different codec to support large set of platform with heigher QoE. Unlike TV broadcasting, live streaming are mostly crowd sourced. The crowd source streamer and viewers are geo distributed. The streaming service provider need to support all those kinds of streamer and viewer and need to provide best possible QoE to them. Chen \etal\ explored the possibility of locate suitable transcoding server as per the geo location of streamer and viewer with the help of cloud federation in their research paper \cite{7218642}.

Even though cloud can be used to transcode live streaming effectively, it not always economical as not all the videos are equally popular. However, transcoding in multiple codec and quality is high. So streaming provider need to be careful about selection of codec and quality level. Aparicio-Pardo \etal\ design an ILP based algorithm in their paper \cite{10.1145/2713168.2713177} to decide the number of quality level require for a live streaming based on the streaming charactics and popularity. It decides CPU budget for each live stream and based on that it decides the quality levels.

\subsection{Video delivery network (VDN)}
Video delivery network is specifically designed CDN for live video streaming proposed by Mukherjee \etal\ \cite{10.1145/2785956.2787475}. \red{Problem with the motivation} 



\subsection{SmoothCache 2.0}
To reduce the distribution overhead, server load and to improve the quality of experience during the live streaming, SmoothCache 2.0\cite{10.1145/2713168.2713182} provide a solution involving peer to peer networking with DASH. Roverso \etal\ exploit the fact that in case of live streaming all the player need to be at sync with very little tollerance $\delta$. The $\delta$ is the time when a player search for the require segment in other peer and if failed fetch it from the CDN. The authors uses several optimization to reduce the overhead like pro-active prefetching. Pro-active prefetching allow few players to fetch segment before as soon as they are available at the server.

\subsection{LiveNAS}
QoE of a live stream is dependent on the network quality and computation capability, especially when random individual user start streaming using comodity devices like smartphone with cellular data connection. Due to lack of dedicated Internet connection and lack infracture, these user can not ensure a steady streaming quality. Kim \etal\ design LiveNAS\cite{10.1145/3387514.3405856} to improve video quality in such cases. LiveNAS runs super resolution to upgrade video quality at the ingest server.

