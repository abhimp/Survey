
\section{Live}
DASH-like video streaming system is widely used as it is superior to other systems, and live video streaming is no exception. Although interactive video sessions are difficult to support via DASH due to the rigid latency bound, non-interactive live video streamings are well supported. Static (or VoD) and live video streaming are almost the same except that video segments are not available in the system in the case of live streaming. All the segments need to capture, processed, and distribute to the nearest server before it can be served to the client. This small difference invite lot of innovation towards DASH-based live streaming. We will discuss a few of those innovations in this section.

\subsection{Crowdsourced Live Streaming}
Live video streaming requires live transcoding into multiple different quality versions and different codecs to support a large set of platforms with higher QoE. Unlike TV broadcasting, live streamings are mostly crowdsourced. The crowdsource streamer and viewers are geo-distributed. The streaming service provider needs to support all those kinds of streamers and viewers and provide the best possible QoE. Chen \etal\ explored the possibility of locating a suitable transcoding server as per the geolocation of streamer and viewer with the help of cloud federation in their research paper \cite{7218642}.

Even though the cloud can be used to transcode live streaming effectively, it not always economical as not all the videos are equally popular. However, transcoding in multiple codec and quality is high. So streaming providers need to be careful about the selection of codec and quality level. Aparicio-Pardo \etal\ design an ILP based algorithm in their paper \cite{10.1145/2713168.2713177} to decide the number of quality levels required for live streaming based on the streaming characters and popularity. It decides the CPU (GPU) budget for each live stream and decides the quality levels. 


\subsection{SmoothCache 2.0}
To reduce the distribution overhead, server load and to improve the quality of experience during the live streaming, SmoothCache 2.0\cite{10.1145/2713168.2713182} provide a solution involving peer to peer networking with DASH. Roverso \etal\ exploit the fact that all the live streaming players need to be in sync with very little tolerance $\delta$. The $\delta$ is the time when a player searches for the required segment in other peers and, if failed, fetch it from the CDN. The authors use optimizations like pro-active prefetching to reduce the overhead. Pro-active prefetching allow few players to fetch segment before as soon as they are available at the server.

\subsection{Neural-Enhanced Live Streaming (LiveNAS)}
QoE of a live stream is dependent on the network quality and computation capability, especially when random individual users start streaming using commodity devices like a smartphone with a cellular data connection. Due to the lack of dedicated Internet connection and lack of infrastructure, these users can not ensure a steady streaming quality. Kim \etal\ design LiveNAS\cite{10.1145/3387514.3405856} to improve video quality in such cases. LiveNAS runs super-resolution to upgrade video quality at the ingest server.

\subsection{Summary}
Live video streaming is apparently similar to the VoD video streaming system. However live streaming need server side support for live content generation and distribution. This task fairly complex if the a streaming provider wants to support crowdsourced live streaming. They have to allocate resources to encode and distribute video content. On the other hand, live streaming of mega event attract millions of viewer globally. Existing literature tries to scale use distribution network by utilizing P2P network. Work like \cite{10.1145/2713168.2713182} tries to utilies peer-to-peer network to share video segments among player. However it may not work well if player have to use the Internet uplink to share the video segment as uplink speeds are usually low and it counts towards there data cap. We believe there is scope to improve live streaming as lot of live streaming video player share same Internet connection.