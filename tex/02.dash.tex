\section{Dynamic Adaptive Streaming over HTTP}
In 2009, Apple developed the HTTP Live Streaming to replace the existing RTSP based streaming for its QuickTime live streaming server\footnote{\url{https://appleinsider.com/articles/09/07/08/apple_launches_http_live_streaming_standard_in_iphone_3_0.html} (accessed: \today)}. Dynamic Adaptive Streaming over HTTP (DASH), sometime call as MPEG-DASH is a technology developed under MPEG to stream video over HTTP. MPEG started working for DASH in 2010 and standardized in 2011\cite{ISO/IEC23009-1:2019}. After release of {\tt dash.js} by DASH Industry Forum (DASH-IF), almost all the the streaming streaming services including YouTube, NetFlix, samsung adopted the DASH in their services.

The DASH standard is simple. To stream video using DASH like system, any HTTP file server can be used. All the files need to be stored in the server. A DASH based video steaming system required three type of files, i) media index files, ii) media data files, and iii) a media presentation description file.

{\bf Media index files:} In general, when any media data stored in a file, the file contain two informations, a) information regarding the media stored in the file and b) the encoded media data. Information regarding media contain all the information to decode the encoded data which is kind of index for the original media data. The DASH mandate that this index data should store in a separate file. This files are usually smaller than any of the Media data, so it does not pose a overhead to the system.

{\bf Media data files:} The media data which actually contain the information to reproduce the media stored in the separate. Here DASH suggest that this media data should be segmented in small multiple chunk with equal playback duration. DASH also suggest that a chunk should be self sufficient and a player can play the any chunk if it have the index file.

{\bf Media presentation description:} It is a {\tt XML} file that contain the meta data required to stream the video. It lists the different audio and video quality available for stream and the URLs for the media index and media data. It also contain the codec, bitrate and playback duration of individual chunk for each quality.

Whenever a player wants to play the DASH based video streaming, it first have to download the MPD file and read it to get other information regarding the stream. Then the player have to decide which quality it want play based on the ABR algorithm running inside the player and get the chunk of the preferred quality accordingly. DASH offloads entire decision to the player so that any existing content delivery network can be used to stream video content. There are instances where media index files and media data are not chunked rather kept in a fixed file. While this saved lot of disk space, the server have to support the HTTP range-request\footnote{\url{https://developer.mozilla.org/en-US/docs/Web/HTTP/Range_requests}}, and player can get the desired chunk using the {\tt range} header.

The simplicity of the DASH lies in the offloading of the computation to the player instead of the server. So, any existing webserver can serve video stream without any modification. Also, the use of MPD file allow DASH streaming server to support large number of the player without any hassle. Service provider can encode the video and audio in all the codec it wants to support and create MPD file containing all the different type of codec as well as quality levels. The player have to choose the codec it can play from the MPD file.

As MPD file can contain multiple media code as well as the multiple quality version of each codec. The player have to be smart smart and choose correct codec and the correct quality to play a video. As per DASH standard, the players are smart enough not only to choose the correct codec and quality for playback but also to change the quality during playback to reduce the rebuffering. A DASH player run adaptive bitrate (ABR) algorithm to decide when to change quality.

\subsection{Adaptive Bitrate Algorithm}
The adaptive bitrate algorithm is the heart of the DASH based streaming system as it decide the quality of each and every segment on the fly. By default, ABR algorithm runs just before fetching the next chunk. As ABR algorithm is part of the player, it have access to all the playback related parameters and it can use any of those parameter to decide quality for next chunk. The primary goal of any ABR algorithm is to maximize to QoE which is combination of three parameters:
\begin{itemize}
	\item {\bf Quality:} Quality of video or audio is determined by the sharpness of the video. Sharpness of a video is directly related to the bitrate of sharpness increases. It is expected that sharp media have better details in the media in both audio and video thus provide better experience in enjoying the stream. Any ABR tries to maintain bitrate as high as possible so that the quality of experience improves.
	\item {\bf Smoothness:} DASH allows player to change video quality on the fly. Frequent quality change disrupt the smoothness of the video watching experience. So, ABR need to minimize the quality change during playback.
	\item {\bf Rebuffering:} Rebuffering is the most irritating experience for any user. The entire DASH and DASH like system is developed just to minimize rebuffering. It outmost responsibility of any ABR to minimize the rebuffering by selecting appropriate quality based of the current network quality.
\end{itemize}

The ABR algorithms are part of the DASH based streaming client. Any player have to implement at least one ABR algorithm in there implementation to support the quality adaptation.

The penetration of the smartphone and wide access to 4G-LTE cellular data is one of the many reason behind the popularity of DASH based streaming. All the major smartphone provide API to build DASH based streaming client using native player. Most of the smartphones also support HTML5 enable webview using which DASH based player can be developed at ease.

\subsection{HTML5 and DASH}
The world wide web consortium (W3C) specifies the experimental media source extension (MSE) which allow JavaScript to manuplate HTML5 video player directly. It also allows to push byte-stream of media data directly to the player decoder and change the player buffer as per requirement on-the-fly. The MSE provide a rich set of API which is being used to developed DASH based video streaming client as all the major browser support the MSE. With introduction of MSE, providing video streaming service become easier as no fancy player need to be installed in the user end and it become easier to update the player just by changing the {\tt JavaScript} player library.
