\section{YouTube}
YouTube started as a video sharing service in 2005. Later, it was acquired by Google. At the inception YouTube used to use the Adobe Flash Player plugin to play video in the browser. At this stage YouTube used to use the Flash Video format for playing video in the browser. However, it allowed user to upload video in various format including WMV, MPEG, AVI. YouTube also exploit the progressive download feature from of Adobe Flash Player to play the video with partially downloaded file\cite{10.1145/1298306.1298310}. After the launch of HTML5 standard, YouTube started using HTML5 embedded video player as an experimental version in January 2010. From the inception, YouTube used to use the fixed resolution (i.e. not adaptive) video playback with a option to change video quality manually.
In 2013, YouTube started trial on DASH based streaming in YouTube and make it the default playback mechanism in the 2015\footnote{\url{https://arstechnica.com/gadgets/2015/01/youtube-declares-html5-video-ready-for-primetime-makes-it-default} (accessed: \today)}.

\subsection{Existing literature}
Existing studies on YouTube streaming and quality of experience (QoE) can be grouped into two broad classes. The first class of works explore traffic patterns and video QoE properties of YouTube~\cite{gill2007youtube,krishnappa2013dashing,wamser2016modeling,wamser2015poster,6757893ieeeexp,7129790ieeeexp}. These papers mostly study YouTube behavior at the periphery, which although provides a summary of performance metrics, but fails to say much about the internals of YouTube's video streaming protocol. The second class of studies, however, explore adaptive streaming characteristics of YouTube. In \cite{finamore2011youtube}, the authors investigate YouTube's data delivery system from the end user view, and illustrate evidence of massive wastage of downloaded data, since viewers often do not watch entire videos -- the study, however, was performed at a time when YouTube used progressive download as the streaming mechanism, and is therefore stale. \cite{krishnappa2013dashing} is probably the first work to evaluate YouTube's performance since its adoption of adaptive streaming -- the authors claim that YouTube gains $83\%$-$95\%$ in terms of bandwidth by switching from progressive download to DASH. Some recent works~\cite{sieber2015cost,seufert2015youtube,sieber2016sacrificing} study YouTube's DASH behavior to analyze the trade-off between quality and data wastage -- however, as already pointed out in~\S\ref{sec:introduction}, their approximations lead to gross overestimation. they perform controlled experiments by varying the underlying link bandwidth, and compute wastage.