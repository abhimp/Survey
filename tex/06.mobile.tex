\section{Video streaming to SmartPhone}
With the availability of cheap data plan over LTE network and content availability in local language led to increase in video playback time in smartphone to a record. No doubt that researchers and industries start investing their resources to improve the video streaming experience over smartphone. SmartPhone comes with variety of screen sizes and battery capacity. These variation become goal to provide best possible video quality while draining least amount of battery backup so that viewer can enjoy video to the fullest for a longer time. In this section we discus few of the various technique developed specifically for online video streaming over smartphone.

\subsection{Energy-Aware video streaming}
Video streaming is extremely power hungry service. Prefetching can be used to reduce the power consumption \cite{6681586,10.1145/2079296.2079321}, however it cause lot of data wastage which is not cheap that time. Hu \etal\ proposed a solution to make video streaming energy efficient by On-Off scheme \cite{7218493}. The scheme exploit the energy states of LTE radio which is consist of 3 state a) Active/On (high energy), b) Tail (medium energy) and c) Off (low energy). The jump from Off to On state require promotion energy. No jump from Off to Tail or On to Off is possible. Hu \etal\ suggested that smartphone should have fixed buffer and when the radio state is in On state, app should fill the buffer before it goes to Tail state.

