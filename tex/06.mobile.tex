\section{Video streaming to SmartPhone}
The availability of cheap data plan over LTE network and content available in local language led to an increase in video playback time in smartphones to a record. No doubt that researchers and industries start investing their resources to improve the smartphone's video streaming experience. The smartphone comes with a variety of screen sizes and battery capacity. These variations become the goal to provide the best possible video quality. At the same time, draining the least amount of battery backup so that viewers can enjoy the video to the fullest for a longer time. There are three component consume energy during video streaming over a smartphone. These are a) screen, b) decoding hardware, and c) radio. There are scopes to reduce energy consumption in all three components. Rebuffering increases the screen on time, thus reducing rebuffering reduces the energy requirement by the online video playback. The use of an efficient codec can reduce energy consumption by the decoding hardware and intelligent use of the cellular radio to reduce energy requirement, among these parameters, codec, mostly static, and easy to decide the suitable codec based on the smartphone variant. However, reducing rebuffering and optimizing radio is challenging due to various parameters, including the device's mobility pattern, cell-tower distribution, cell-tower load, etc. This section discusses a few of the various techniques developed specifically for online video streaming over the smartphone.

\subsection{Energy-Aware video streaming}
Video streaming is an extremely power-hungry service. Prefetching can be used to reduce the power consumption \cite{6681586,10.1145/2079296.2079321}. However, it causes a lot of data wastage, which is not cheap at that time. Hu \etal\ proposed a solution to make video streaming energy efficient by On-Off scheme \cite{7218493}. The scheme exploits the energy states of LTE radio, which is consist of 3 state a) Active/On (high energy), b) Tail (medium energy), and c) Off (low energy). The jump from Off to On state requires promotion energy. No jump from Off to Tail or On to Off is possible. Hu \etal\ suggested that the smartphone should have a fixed buffer, and when the radio state is in On state, the app should fill the buffer before it goes to Tail state.

\subsection{Energy consumption by DASH over LTE}
Zhang \etal\ \cite{10.1145/2910018.2910656} measured power consumption by DASH video streaming over LTE. They measured the energy consumption using Monsoon power monitor \cite{monsoonmonitor} tool on various streaming strategies and conditions. The study yields that network-based energy consumption can be reduced up to 30\% just by change segment length from 2 sec to 4 sec. Similarly, increase buffer size can also reduce energy consumption.

\subsection{OSCAR}
OSCAR\cite{10.1145/2910018.2910655} is a hybrid ABR algorithm specially designed for the smartphone to reduce the stall while in mobility. Zahran \etal\ model the throughput as a random variable with Kumaraswamy distribution \cite{jones2009kumaraswamy} to estimate the stall probability. The adaptation technique is similar to the buffer-based ABR algorithms with three thresholds low, transient, and high. At low buffer condition, it just downloads the lowest quality segment to fill the buffer. However, the algorithm tries to avoid radio off state by downloading the high (maximum among highest supported by the estimated throughput and 1level up from the last quality) quality segment. A segment's quality is determined by solving an optimizing problem to maximize the sum of an exponential video utility function and switching penalty. As per the simulation result, the solution yields up to 85\% stall-free playback time.

\subsection{Summary}
Although DASH attracts researcher to study, design, and develop ABR algorithm, streaming solution to improve QoE. However very few of them concerned about the energy consumption by online video streaming on smartphone. We discuss few such research in this section. Some of these researches, studied the energy consumption\cite{10.1145/2910018.2910656} and few provides algorithm to improve power efficiency\cite{10.1145/2910018.2910655}. However, there is no solution to provide energy efficient video stream solution while the user is mobile.