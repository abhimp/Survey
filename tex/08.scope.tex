\section{Summary and open scopes}
The online video streaming system is a vast and complex system to provide the best QoE for all the users and efficient energy consumption for smartphone-based users. In this chapter, we discussed various research works and studies to improve the video streaming system. Now we summarize them and discuss the scope of this topic.

\subsection{YouTube}
YouTube started as a video sharing service in 2005. Later, it was acquired by Google. At the inception, YouTube used the Adobe Flash Player plugin to play video in the browser. Although YouTube played video in the browsers only using the Flash Video format, it allowed users to upload video in various formats, including WMV, MPEG, AVI. YouTube also exploited the progressive download feature from Adobe Flash Player to play the video with partially downloaded file\cite{gill2007youtube}. After the HTML5 standard launch, YouTube started using HTML5 embedded video player as an experimental version in January 2010. From the inception, YouTube used to use the fixed resolution (i.e., not adaptive) video playback with an option to change video quality manually.
However, in 2013, YouTube started the trial on DASH based streaming in YouTube and made it the default playback mechanism in the 2015\footnote{\url{https://arstechnica.com/gadgets/2015/01/youtube-declares-html5-video-ready-for-primetime-makes-it-default} (accessed: \today)}.

Existing studies on YouTube video streaming system and video QoE can be grouped into two broad classes. The first class of works explore traffic patterns and video QoE properties of YouTube~\cite{gill2007youtube,krishnappa2013dashing,wamser2016modeling,wamser2015poster,6757893ieeeexp,7129790ieeeexp}. These works mostly study YouTube's behavior at the periphery, which, although provides a summary of performance metrics, fails to say much about the internals of YouTube's video streaming protocol. The second class of studies, however, explore the adaptive streaming characteristics of YouTube. In \cite{finamore2011youtube}, the authors investigated YouTube's data delivery system from the end-user view. They illustrated evidence of massive wastage of downloaded data since viewers often do not watch entire videos -- the study, however, was performed at a time when YouTube used progressive download as the streaming mechanism and is therefore stale. \cite{krishnappa2013dashing} is probably the first work to evaluate YouTube's performance since its adoption of adaptive streaming -- the authors claim that YouTube gains $83\%$-$95\%$ in terms of bandwidth by switching from progressive download to DASH. Some recent works~\cite{sieber2015cost,seufert2015youtube,sieber2016sacrificing} study YouTube's DASH behavior to analyze the trade-off between quality and data wastage -- however, their approximations lead to gross overestimation. They perform controlled experiments by varying the underlying link bandwidth and compute wastage.


\subsection{Effect of Transport Protocols and Device Mobility}
Major smartphone users watch online videos while commuting, and video streaming services are pretty power hungry services. So, providing a power-efficient solution for the smartphone user who watches online videos while in mobility in a dire need.

DASH or DASH-like video streaming systems use HTTP(S) protocol to fetch content from the server. While HTTP(S) is widely accepted, it uses TCP as the underlying protocol. So, the performance of DASH based video streaming system is highly dependent on the performance of TCP. Although TCP's performance is best for long running flows, it does not work well with short flows. Furthermore, most of the ABR system produce ON-OFF traffic for DASH based streaming system. ON-OFF traffics are essentially short flows. So, DASH can not perform at its peak due to this effect. To avoid such a problem, Google developed a new transport protocol QUIC\cite{langley2017quic} which works on top of UDP and provides an interface for HTTPS. According to Google, QUIC reduces rebuffering up to 15\% for mobile users.

\subsection{Impact of QUIC on DASH}
Various recent studies \cite{Biswal2017,Megyesi2016,bhat2017not} have revealed that QUIC can improve web performance by reducing the page load time even at poor network conditions. Among them, a few works have explored the adaptive streaming performance over QUIC. In~\cite{bhat2017not}, the authors have empirically shown that DASH suffers over QUIC. Although they have given the first indication that the current ABR techniques might not perform well over QUIC, their analysis is mostly focused on buffer-based ABR and does not look into various QoE metrics as explored in the recent literature~\cite{yin2015control,mao2017neural}. In a follow-up work~\cite{bhat2018improving}, the authors have explored the QUIC retransmissions to improve the buffer-based ABR over DASH. In~\cite{van2018empirical}, the authors have used an emulated setup to analyze the buffer-based ABR techniques over QUIC and proposed a QoE prediction mechanism for adaptive streaming over QUIC. Also, these existing studies have indicated that QUIC might not suit well for buffer-based ABR. They have not explored the performance of advanced ABR techniques over QUIC, although it is important as QUIC is gaining popularity and has become a choice of protocol for most web services. 
These reasons led us to believe that there is a requirement for an in-depth study on the performance of different advanced ABR algorithm on top of QUIC.

\subsection{Smartphone Energy consumption and DASH based video streaming}
Power consumption is a concern with all mobile devices, and smartphones are no different. While most of the services like cellular, BlueTooth, push notifications are designed to consume less energy to provide long battery backup. However, online video streaming services are not part of those services, and it is challenging to improve significant energy efficiency. As we discussed before, the existing work has just scratch the surface of the energy-efficient solution for video streaming services. So, there are scopes to improve the energy-efficiency of online video services. To understand those scopes, we perform a study in various cities in India and presented the result in section~\S\ref{sec:chap03:DASHinMobility}. With the outcome of our study and existing literature, we try to develop an energy-efficient video streaming system. We describe our algorithm in chapter~\S\ref{chapter04}.

\subsection{DASH and live video streaming system}
Online live video streaming is becoming an alternative to the live TV broadcast. Online live streaming services stream the big and important events and also allow individual users to go live and broadcast videos. DASH supports live streaming as the playback technique is the same for both live and VoD streaming. However, the server-side part is not the same, and there are several challenges to serve live streaming to a very large audience from all over the globe. It is very difficult to serve live stream to millions of users even with CDN. Live streaming providers are trying to exploit live streaming's synchronous behavior (i.e., all the players play almost the same portion of the video) and extend the CDN using P2P networks. While most of the work concentrates on routing P2P traffic through ISP, there is a scope to exploit the same for in campus networks (i.e., lot player connected via private network and common Internet backbone). In chapter~\S\ref{chapter06}, we try to investigate this scope and develop a solution.