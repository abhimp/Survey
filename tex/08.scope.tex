\section{Summary and open scopes}
The online video system is vast and very complex to provide a perfect solution. In this chapter we discussed several research work to improve the video streaming system. Now we summerise the research work in a whole and discuss the scope about those work.

\subsection{YouTube}
YouTube started as a video sharing service in 2005. Later, it was acquired by Google. At the inception YouTube used to use the Adobe Flash Player plugin to play video in the browser. At this stage YouTube used to use the Flash Video format for playing video in the browser. However, it allowed user to upload video in various format including WMV, MPEG, AVI. YouTube also exploit the progressive download feature from of Adobe Flash Player to play the video with partially downloaded file\cite{gill2007youtube}. After the launch of HTML5 standard, YouTube started using HTML5 embeddSacrificing efficiency for quality of experienced video player as an experimental version in January 2010. From the inception, YouTube used to use the fixed resolution (i.e. not adaptive) video playback with a option to change video quality manually.
In 2013, YouTube started trial on DASH based streaming in YouTube and make it the default playback mechanism in the 2015\footnote{\url{https://arstechnica.com/gadgets/2015/01/youtube-declares-html5-video-ready-for-primetime-makes-it-default} (accessed: \today)}.

Existing studies on YouTube streaming and quality of experience (QoE) can be grouped into two broad classes. The first class of works explore traffic patterns and video QoE properties of YouTube~\cite{gill2007youtube,krishnappa2013dashing,wamser2016modeling,wamser2015poster,6757893ieeeexp,7129790ieeeexp}. These papers mostly study YouTube behavior at the periphery, which although provides a summary of performance metrics, but fails to say much about the internals of YouTube's video streaming protocol. The second class of studies, however, explore adaptive streaming characteristics of YouTube. In \cite{finamore2011youtube}, the authors investigate YouTube's data delivery system from the end user view, and illustrate evidence of massive wastage of downloaded data, since viewers often do not watch entire videos -- the study, however, was performed at a time when YouTube used progressive download as the streaming mechanism, and is therefore stale. \cite{krishnappa2013dashing} is probably the first work to evaluate YouTube's performance since its adoption of adaptive streaming -- the authors claim that YouTube gains $83\%$-$95\%$ in terms of bandwidth by switching from progressive download to DASH. Some recent works~\cite{sieber2015cost,seufert2015youtube,sieber2016sacrificing} study YouTube's DASH behavior to analyze the trade-off between quality and data wastage -- however their approximations lead to gross overestimation. they perform controlled experiments by varying the underlying link bandwidth, and compute wastage.


\subsection{Effect of Transport Protocols and Device Mobility}
Penetration of LTE cellular network and smartphone is one of the reason that the online video streaming is the most popular service in the Internet. However smartphones are mobile and there is large number of users watch online video while commute. Also, the video streaming services are pretty power hungry service. So, providing the a power efficient solution for the smartphone user who watch online video while in mobility.

DASH or DASH-like video streaming system use HTTP(S) protocol to fetch content from the server. While HTTP(S) is widely acceptable, HTTP(S) uses TCP as the underlying protocol. So, the performance of DASH based video streaming system is highly dependent on the TCP. While TCP perform well for long running flows, it does not work well with short flows. However, most of the ABR system produce ON-OFF traffic for DASH based streaming system. ON-OFF traffic are essentially short flows. So, DASH can not perform at its peak due to this effect. To avoid such problem Google developed a new transport protocol QUIC\cite{langley2017quic} which work on top of UDP and provide a interface for HTTPS. According Google's claim QUIC reduces rebuffering up to 15\% for mobile users.

\subsection{Impact of QUIC on DASH}
Various recent studies \cite{Biswal2017,Megyesi2016} have revealed that QUIC can improve the web performance by reducing the page load time even at poor network conditions. Among them, a few works have explored the adaptive streaming performance over QUIC. In~\cite{bhat2017not}, the authors have empirically shown that DASH suffers over QUIC. Although they have given the first indication that the current ABR techniques might not perform well over QUIC, their analysis is mostly focused on buffer-based ABR and does not look into various QoE metrics as explored in the recent literature~\cite{yin2015control,mao2017neural}. In a follow-up work~\cite{bhat2018improving}, the authors have explored the QUIC retransmissions to improve the buffer-based ABR over DASH. In~\cite{van2018empirical}, the authors have used an emulated setup to analyze the buffer-based ABR techniques over QUIC and also proposed a QoE prediction mechanism for adaptive streaming over QUIC. Also, these existing studies have indicated that QUIC might not suit well for buffer-based ABR. They have not explored the performance of advanced ABR techniques over QUIC although it is important as QUIC is gaining popularity and become a choice of protocol most of the web-services. For these reason we perform a study on the performance of different advanced ABR algorithm on top of QUIC. In our study we also try to find out the root cause of any performance penalty.

\subsection{Smartphone Energy consumption and DASH based video streaming}
Online video streaming is a power hungry service. However, online video streaming is one of the most popular service and and every like to enjoy it. So, efficient video streaming system for smartphone may not be the system which can provide highest QoE instead the one which can provide reasonable QoE and saves lot of energy. Prefetching video while network condition is good and play it offline(\cite{6681586,10.1145/2079296.2079321}) is probably the best solution. However it is infeasible to deploy. Instead exploiting radio energy states are more efficient an feasible (\cite{7218493}). It is a challenge to find optimal scheduling to exploit the energy states of cellular networks. To find a solution first thing require is to measure the power consumption pattern by DASH based video streaming system and then one improved system can be designed with the information/ Existing work such as \cite{10.1145/2910018.2910656} found out that there are scopes to reduce energy consumption by smartphone while video streaming. To understand those scope, we perform a study on various cities in India and presented the result in the section \S\ref{sec:chap03:DASHinMobility}. With the outcome of our study and existing literature we try to develop a power efficient video streaming system. We describe our algorithm in the section \S\ref{chapter04}.

\subsection{DASH and live video streaming system}
Online live video streaming becoming alternate to the live TV broadcast. Online live streaming not only stream the big and important event but also allow individual user to go live and broadcast videos. DASH support live streaming as the playback technique is same for both live and VoD streaming. However, a streaming service provider need to consider several hardware and software dependancy to support live video streaming. They need adequet amount to computing power to encode incoming video in multiple quality level live and distribute them to the client. They have allocate resources in such a way so that viewer gets the maximum QoE and support all the live streaming with minimal expenses. In case to very popular videos, number viewers grows exponentially globally distributed. It is very difficult to scale such number of viewers even with the help of CDN. So, providers are trying to find alternate solution to ease the job. They are trying to extend the CDN with the help of peer-2-peer network. The advantage of the live streaming is that all the players are in sync. So, every player watch exactly same (live) part of the video. So, one player can easily share with other. However, it is not simple as ISPs are not friendly to clients communicate with a client from another ISP. Also, peer-2-peer delay is high. So, SmoothCache 2.0 try to form cluster among the users from same ISP. There are other work like \cite{10.1145/3387514.3405856,9155467,8057140} done to extend the CDN with the help of p2p network. There goals are different. All though there are several work exist to extend the CDN with P2P network, none them directly exploit the fact that users some concentrate in side a single campus or office where they are connected via unrestricted LAN connection. In chapter \S\ref{chapter06} we try to exploit these phenomenon.